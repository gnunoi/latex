%!TEX program = xelatex
<<<<<<< HEAD
\documentclass[t]{ctexbeamer} %[t]:正文顶端对齐
=======
\documentclass[aspectratio=169]{ctexbeamer}
>>>>>>> 83c71ceec8a941d3ba155816bedbebfad4342ae2
\usepackage{geometry}
%\geometry{paperwidth=320mm, paperheight=180mm}
\usepackage{ctex}
\usepackage{tikz}
\usepackage{amsmath}

\usepackage{fontspec}
\setmainfont{Times New Roman} %设置默认衬线字体
\setsansfont{Arial} %设置默认无衬线字体
\setmonofont{Courier New} %设置等宽字体

\setbeamerfont{title}{size=\fontsize{60}{72}\selectfont} %设置title的字体字号
\setbeamerfont{subtitle}{size=\fontsize{36}{48}\selectfont} %设置subtitle的字体字号
\setbeamerfont{author}{size=\fontsize{24}{36}\selectfont} %设置author的字体字号
\setbeamerfont{date}{size=\fontsize{24}{36}\selectfont} %设置date的字体字号
\setbeamerfont{frametitle}{size=\fontsize{36}{48}\selectfont} %设置frametitle的字体字号
\setbeamerfont{framesubtitle}{size=\fontsize{20}{30}\selectfont} %设置framesubtitle的字体字号
\setbeamerfont{definition}{size=\fontsize{36}{48}\selectfont} %设置definition的字体字号
\setbeamerfont{problem}{size=\fontsize{36}{48}\selectfont} %设置problem的字体字号
\setbeamertemplate{normal text}{\fontsize{24}{36}\selectfont} % 调整正文字号

\begin{document}
\title{代数习题讲解}
\subtitle{为未知而教,为未来而学}
\author{丁保华}
\date{\today}
\maketitle

\begin{frame}[t]
\frametitle{目录}
\tableofcontents
\end{frame}

\section{数列}
\begin{frame}
\frametitle{斐波那契数列(Fibonacci sequence)}
\framesubtitle{由意大利数学家莱昂纳多·斐波那契(Leonardo Fibonacci)在1202年提出。 \textit{又称:兔子数列。}}
\vspace{1cm}
\begin{definition}
\[
\begin{aligned}
&F(0)=0 \\
&F(1)=1 \\
&F(2)=2 \\
&\cdots \\
&F(n)=F(n-1)+F(n-2)
\end{aligned}
\]
\end{definition}

\begin{examples}
\[
\begin{aligned}
&1, 2, 3, 5, 8, \cdots
\end{aligned}
\]
\end{examples}

\end{frame}


\section{习题}
\begin{frame}
\frametitle{习题1.1及讲解}
\begin{problem}
\[
x^2+2x+1=0
\]
\end{problem}

\end{frame}



\end{document}
%!TEX program = xelatex
\documentclass{article}
\usepackage{amsmath}
\usepackage{ctex}
\title{小学五年级奥数试卷}
\author{}
\date{}

\begin{document}

\maketitle

\section*{一、选择题}(每题5分,共25分)

\begin{enumerate}
    \item 在一个班级里,30名学生参加了运动会。每位学生参加1项或2项运动。已知有18人参加了跑步比赛,15人参加了跳远比赛,10人既参加跑步又参加跳远比赛。那么,参加跳远比赛但没有参加跑步比赛的学生人数是多少?
    \begin{itemize}
        \item A. 5人 \hspace{1cm} B. 10人 \hspace{1cm} C. 8人 \hspace{1cm} D. 7人
    \end{itemize}
    
    \item 小明去超市买了一袋苹果和一袋橙子,苹果和橙子的总重是5千克。苹果的重量是橙子的3倍,问苹果和橙子的各自重量是多少?
    \begin{itemize}
        \item A. 苹果3千克,橙子2千克 \hspace{1cm} B. 苹果4千克,橙子1千克 \hspace{1cm} C. 苹果2千克,橙子3千克 \hspace{1cm} D. 苹果3.5千克,橙子1.5千克
    \end{itemize}
    
    \item 1, 4, 9, 16, 25, ? 下一个数应该是什么?
    \begin{itemize}
        \item A. 30 \hspace{1cm} B. 36 \hspace{1cm} C. 32 \hspace{1cm} D. 40
    \end{itemize}
    
    \item 一块长方形的纸,长为12厘米,宽为5厘米。将这块纸剪成两个相同的正方形,剩下的部分是什么形状?
    \begin{itemize}
        \item A. 长方形 \hspace{1cm} B. 正方形 \hspace{1cm} C. 梯形 \hspace{1cm} D. 三角形
    \end{itemize}
    
    \item 有一个四边形的周长是40厘米,如果两个边长分别是10厘米和12厘米,另外两个边长分别是多少厘米?
    \begin{itemize}
        \item A. 9厘米,9厘米 \hspace{1cm} B. 8厘米,10厘米 \hspace{1cm} C. 11厘米,9厘米 \hspace{1cm} D. 8厘米,12厘米
    \end{itemize}
\end{enumerate}

\section*{二、填空题}(每题4分,共20分)

\begin{enumerate}
    \item 一个正方形的边长是6厘米,它的面积是\underline{\hspace{2cm}}平方厘米。
    \item 一个班级里有40名学生,男生人数是女生的2倍,女生有\underline{\hspace{2cm}}人。
    \item 小华和小明有一些糖果,小华比小明多6颗糖果。如果小华有12颗糖果,那么小明有\underline{\hspace{2cm}}颗糖果。
    \item 一个圆的直径是10厘米,它的半径是\underline{\hspace{2cm}}厘米。
    \item 小强从家走到学校,走了150米后停下来休息,然后继续走了200米,最终到达学校。小强一共走了\underline{\hspace{2cm}}米。
\end{enumerate}

\section*{三、解答题}(每题10分,共30分)

\begin{enumerate}
    \item 有6个苹果和4个橙子,小明要将它们平均分给3个朋友。每个朋友能分到多少个苹果和橙子?
    
    \item 如果一个数除以4余3,除以5余2,除以6余1,这个数最小是多少?
    
    \item 小红和小刚一起去买文具。小红买了3本书和5支笔,花了34元;小刚买了4本书和3支笔,花了31元。书和笔的价格分别是多少?
\end{enumerate}

\section*{四、综合题}(15分)

在一个棋盘上,每个小方格的边长是1厘米。现在有一个正方形的区域,边长为6厘米,这个区域包含若干个小方格。问:这个正方形区域的面积是多少平方厘米?它包含的单元格数是多少个?

\section*{参考答案}

\begin{enumerate}
    \item A
    \item A
    \item B
    \item A
    \item A
\end{enumerate}

填空部分答案:
\begin{enumerate}
    \item 36
    \item 13
    \item 6
    \item 5
    \item 350
\end{enumerate}

解答题参考:
\begin{enumerate}
    \item 每个朋友能分到2个苹果和3个橙子。
    \item 这个数最小是23。
    \item 书的价格是4元,笔的价格是3元。
\end{enumerate}

综合题参考:
面积是36平方厘米,包含36个小方格。

\end{document}

\documentclass[aspectratio=169]{ctexbeamer} %[t]:顶端对齐
\usetheme{Madrid} %Madrid,蓝色调为主。
\usecolortheme{beaver} %beaver
\usefonttheme{professionalfonts}

\usepackage{universe}
\uBigPaper

\date{\today}
\begin{document}

\begin{frame}[t]{数学课程的总目标}
\begin{spacing}{1.2} %设置1.5倍行距
{\large
通过义务教育阶段的数学学习,学生逐步:\\
\begin{enumerate}[label={\arabic*.}]
\item \textbf{会用数学的眼光观察现实世界;}\\
\item \textbf{会用数学的思维思考现实世界;}\\
\item \textbf{会用数学的语言表达现实世界。}\\
\end{enumerate}
(简称“三会”) 。\\
}
\end{spacing}
\end{frame}

\begin{frame}[t]{数学考试丢分的四大原因}
\begin{spacing}{1} %设置行距
{\large
\begin{enumerate}[label={\arabic*.}]
\item \textbf{知识点不透彻;}\\
\item \textbf{题型不熟练;}\\
\item \textbf{计算不准确;}\\
\item \textbf{计算速度慢。} \\
\end{enumerate}
(简称“四因”) 。\\
}
\end{spacing}
\end{frame}

\begin{frame}[t]{学好数学的五个步骤}
\begin{spacing}{1} %设置行距
{\large
\begin{enumerate}[label={\arabic*.}]
\item \textbf{发现个案(发现有趣的个案);}\\
\item \textbf{类似案例(寻找类似的案例);}\\
\item \textbf{总结规律(找到一般的规律:从特殊到一般);}\\
\item \textbf{定义证明(给出定义或证明)。} \\
\item \textbf{实际应用(应用到实践中去:从一般到特殊)。} \\
\end{enumerate}
(简称“五步骤”) 。\\
\alert{第一步到第三步:大胆假设;第四步:小心求证;第五步:放心应用。}\\
}
\end{spacing}
\end{frame}

\end{document}
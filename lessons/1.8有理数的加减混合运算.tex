\documentclass[aspectratio=169]{ctexbeamer} %[t]:顶端对齐
\usetheme{Madrid} %Madrid,蓝色调为主。
\usecolortheme{beaver} %beaver
\usefonttheme{professionalfonts}

\usepackage{universe}
\uBigPaper

\date{\today}
\begin{document}

\begin{frame}[t]{1.8 有理数的加减混合运算}
\begin{spacing}{1.5}
\large
\begin{enumerate}[label={\arabic*.}]
\item 加减法是一级运算,优先级最低;
\item 加法与减法互为逆运算,加法与减法带符号统一理解为加法;
\item 减一个数,等于加相反数:$a - b = a + (-b)$或$a - (-b) = a + b$;
\item 加一个数,等于减相反数:$a + b = a -(-b)$或$a + (-b) = a - b$;
\item 同号取正(去括号,取正号) \\
$+(+a) = +a = a$ \\
$-(-a) = +a = a$ 
\item 异号取负(去括号,取负号)\\
$-(+a) = -a$ \\
$+(-a) = -a$ \\
\item 加法具有交换律:$a + b + c = a + c + b$;
\item 加法具有结合律:$(a + b) + c = a + (b + c)$。
\end{enumerate}

\end{spacing}
\end{frame}

\end{document}